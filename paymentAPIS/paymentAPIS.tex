\documentclass[a4paper,12pt]{article}

% Packages
\usepackage[utf8]{inputenc}
\usepackage{hyperref}
\usepackage{listings}
\usepackage{xcolor}

% Define code style for JSON
\lstdefinelanguage{json}{
    basicstyle=\ttfamily\footnotesize,
    numbers=left,
    numberstyle=\tiny,
    stepnumber=1,
    numbersep=8pt,
    showstringspaces=false,
    breaklines=true,
    frame=single,
    backgroundcolor=\color{gray!10},
    literate=
     *{0}{{{\color{blue}0}}}{1}
      {1}{{{\color{blue}1}}}{1}
      {2}{{{\color{blue}2}}}{1}
      {3}{{{\color{blue}3}}}{1}
      {4}{{{\color{blue}4}}}{1}
      {5}{{{\color{blue}5}}}{1}
      {6}{{{\color{blue}6}}}{1}
      {7}{{{\color{blue}7}}}{1}
      {8}{{{\color{blue}8}}}{1}
      {9}{{{\color{blue}9}}}{1}
      {:}{{{\color{red}:}}}{1}
      {,}{{{\color{red},}}}{1}
      {\{}{{{\color{black}\{}}}{1}
      {\}}{{{\color{black}\}}}}{1}
      {[}{{{\color{black}[}}}{1}
      {]}{{{\color{black}]}}}{1},
}

% Document begins here
\begin{document}

\title{\textbf{Payment APIs Platform Documentation}}
\author{Your Company Name}
\date{\today}
\maketitle

\tableofcontents
\newpage

% Section 1: Overview
\section{Overview}
This document provides a detailed guide to using the Payment APIs Platform. Our APIs enable developers to integrate seamless payment solutions, including card payments, bank transfers, and mobile payments, into their applications.

\textbf{Features:}
\begin{itemize}
    \item Secure and reliable transactions.
    \item Support for multiple currencies.
    \item Real-time transaction monitoring.
\end{itemize}

\newpage

% Section 2: Getting Started
\section{Getting Started}
To begin using the Payment APIs, follow these steps:

\begin{enumerate}
    \item Sign up for an account at \href{https://www.example.com}{https://www.example.com}.
    \item Obtain your API key from the developer dashboard.
    \item Refer to the endpoint documentation to make your first API call.
\end{enumerate}

\textbf{Base URL:}
\begin{verbatim}
https://api.example.com/v1
\end{verbatim}

\newpage

% Section 3: Authentication
\section{Authentication}
All API requests require an API key for authentication. Include the API key in the header of every request.

\textbf{Header Example:}
\begin{lstlisting}[language=json]
{
    "Authorization": "Bearer YOUR_API_KEY"
}
\end{lstlisting}

\textbf{Authentication Error Response:}
\begin{lstlisting}[language=json]
{
    "error": {
        "code": 401,
        "message": "Unauthorized. Invalid API key."
    }
}
\end{lstlisting}

\newpage

% Section 4: API Endpoints
\section{API Endpoints}

\subsection{1. Create a Payment}
\textbf{Endpoint:}
\begin{verbatim}
POST /payments
\end{verbatim}

\textbf{Request Body:}
\begin{lstlisting}[language=json]
{
    "amount": 5000,
    "currency": "USD",
    "payment_method": "card",
    "description": "Purchase of goods"
}
\end{lstlisting}

\textbf{Response:}
\begin{lstlisting}[language=json]
{
    "id": "pay_123456",
    "status": "success",
    "amount": 5000,
    "currency": "USD"
}
\end{lstlisting}

\subsection{2. Retrieve Payment Details}
\textbf{Endpoint:}
\begin{verbatim}
GET /payments/{payment_id}
\end{verbatim}

\textbf{Response:}
\begin{lstlisting}[language=json]
{
    "id": "pay_123456",
    "status": "success",
    "amount": 5000,
    "currency": "USD",
    "created_at": "2025-01-05T12:00:00Z"
}
\end{lstlisting}

\newpage

% Section 5: Error Codes
\section{Error Codes}
Here are some common error codes and their meanings:

\begin{tabular}{|l|l|}
\hline
\textbf{Error Code} & \textbf{Description} \\ \hline
400 & Bad Request. The request is invalid. \\ \hline
401 & Unauthorized. Invalid API key. \\ \hline
404 & Not Found. The resource does not exist. \\ \hline
500 & Internal Server Error. Try again later. \\ \hline
\end{tabular}

\newpage

% Section 6: Support
\section{Support}
If you have any questions or encounter issues, please contact our support team:
\begin{itemize}
    \item Email: \href{mailto:support@example.com}{support@example.com}
    \item Phone: +1-800-123-4567
    \item Documentation: \href{https://docs.example.com}{https://docs.example.com}
\end{itemize}

\end{document}
